\documentclass{article}
\usepackage{graphicx} % Required for inserting images

\title{M2R Convergence of lattice sum}
\author{group 49}
\date{June 2023}

\begin{document}

\maketitle

\section{Divergence of series}

The usage of Cesaro summation extends beyond its application in handling divergent series and can be used for many different purposes. A finite value can be designated for an untypically converging series through utilization of this method, and the process of computing Cesaro means through this method involves obtaining average values for the first n partial sums.If Cesaro leads us towards any fixed number then that fixed number is actually our series' summation. It's common for Cesaro means to converge more rapidly than those of partial sums thereby providing precise estimates for divergent series.
\\
Consider the series:
\\
1 - 2 + 4 - 8 +16- ...
\\
This particular series fails to converge since its term's absolute value does not approach zero. However, applying Cesaro summation allows us to calculate a finite value for the sum of the series. By taking into account the Cesaro means of our partial sums we are able to do this:
\\
s1 = 1
\\
s2 = (1-2)/2 = -1/2
\\
s3 = (1-2+4)/3 = 1/3
\\
s4 = (1-2+4-8)/4 = -1/2
...

It can be seen that by using the Cesaro method for finding partial sums we reach a conclusion exhibiting convergence towards 1/3. In light of this, we can characterize the series as...

1 - 2 + 4 - 8 + ... = 1/3

This outcome can be quite startling when we consider that there's no steady convergence in terms of partial summations. However, we can use Cesaro summation in order to assign a finite value to the sum of this series.
\\
\newpage
\section{number theory}
Number theorists have been grappling with the classic problem of representing integers as sums of squares for centuries. Our task is to ascertain if it's feasible for us to discover two integers x,y such that their square sums result in the pre-existing positive value n. We can see that while it's possible to write 'For Example' if we factorize 5 as $(1)^{2}+(2)^{2}$, 7 doesn't work in a similar manner

Solving this problem in a methodical manner is achievable by employing the theory of quadratic forms and lattice sums. We can define a quadratic form $Q(x, y)=x \wedge\{2\}+y \wedge\{2\}-n z \wedge\{2\}$ 
\\
by choosing an integral value for z when we have n as a positive integer.
The values of this quadratic form over the integers can be expressed as a lattice sum over the lattice generated by the vectors $(1,0,z)$ and $(0,1,z)$, namely:

$$
\sum_{x,y\in\mathbb{Z}} e^{2\pi i (x^2+y^2-nz^2)} = \sum_{m\in\mathbb{Z}^2} e^{2\pi i (m_1^2+m_2^2)z^2} e^{2\pi i n (m_1,m_2)\cdot(m_1,m_2)}
$$
\\
To get to the value of a lattice sum on RHS expression one needs to add up values computed from quadratic form $m_1^2+m_2^2$ over two plane vectors $(1,0)$ and $(0,1)$.The broader formula for lattice sums over quadratic forms encompasses this specific case known as the Gauss circle problem.
\\
A solution to representing an integer $n$ in form of sum of squares can be found by locating such lattice vector $(m_1,m_2)$ where $m_1^2+m_2^2=n$. Determining the shortest matrix in a given lattice requires analyzing its quadratic formula involving $m_1^2+m_2^2$.  the minimum value of this quadratic form over the lattice is given by using the theory of quadratic forms.
\\
Whenever our quadratic form attains a minimum less than n, we will be able to decompose n into a sum of two or more perfect squares. Counting all vectors having an identical minimum value for their respective quadratic forms will yield the total possible expressions for representing n as a sum of squares. It is possible to connect certain values in modular forms with representations of integers as sums of squares through the so-called theta correspondence.
\\
Therefor, the theory of quadratic forms and lattice sums provides a systematic way to solve the problem of representing integers as sums of squares.
\begin{figure}
    \centering
    \includegraphics[width=0.3\linewidth]{gausspic.jpg}
    \label{gauss circle problem}
\end{figure}




\newpage
\section{Lattice points}
Lattice sums can be used to estimate the number of lattice points in a given region of the plane. Specifically, for a convex region with finite area $\mathcal{L}$, the lattice sum

$$
\sum_{n\in\mathcal{L}} 1
$$

counts the number of lattice points in $\mathcal{L}$. However, this sum can be very difficult to calculate exactly when the number of lattice points in $\mathcal{L}$ is very large.

Instead, by using the convergence of the lattice sum we can obtain an estimate of the number of lattice points in $\mathcal{L}$. Specifically, we can change the lattice sum into other form

$$
\sum_{n\in\mathcal{L}} 1 = \frac{1}{\text{area}(\mathcal{L})} \iint_\mathcal{L} \sum_{m\in\mathbb{Z}^2} e^{2\pi i \langle m, x\rangle} dx
$$

where $\langle m, x\rangle$ denotes the dot product of $m$ and $x$. The inner sum is a lattice sum over all points in $\mathbb{Z}^2$, which converges to a constant value as $m$ grows large. Therefore, we can approximate the lattice sum by truncating the inner sum at some large value $M$:

$$
\sum_{n\in\mathcal{L}} 1 \approx \frac{1}{\text{area}(\mathcal{L})} \iint_\mathcal{L} \sum_{|m_1|,|m_2|\leq M} e^{2\pi i \langle m, x\rangle} dx.
$$

This approximation can be made arbitrarily accurate by choosing a sufficiently large value of $M$. The lattice sum inside the integral can be computed efficiently using the Fast Fourier Transform (FFT), which reduces the computational complexity from $O(M^2)$ to $O(M\log M)$. Therefore, this method provides a practical way to estimate the number of lattice points in a given region $\mathcal{L}$ of the plane.
\end{document}