\documentclass{article}
\usepackage{graphicx} % Required for inserting images

\title{M2R Convergence of lattice sum}
\author{group 49}
\date{June 2023}

\begin{document}

\maketitle

\section{Madelung Constant}
One of those famous applications of lattice sum is Madelung Constant. 
The attraction between the cations and anions in ionic solids is due to their opposite charges.This means that it takes a certain amount of energy to separate them. The separation of anions and cations through breaking their bonding requires a certain quantity of designated power provided by external sources - referred as lattice energy - on each mole of ionic solids under normal settings. Utilizing the concept known as the Madelung constant, we can describe ions in crystals as point charges allowing us to determine their respective electrostatic potentials.

\begin{figure}
    \centering
    \includegraphics[width=0.6\linewidth]{NaCl pic.jpg}
    \label{NaCl structure}
\end{figure}

The Madelung constant enables the computation of the electric	 potential of a single ion in
a crystal.
\begin{equation}
V_i=\frac{e}{4 \pi \epsilon_0 r_0} \sum_j \frac{z r_0}{r_{i j}}=\frac{e}{4 \pi \epsilon_0 r_0} M_i
\end{equation}

where $r_0$ is the nearest neighbour distance and $M_i$ is the (dimensionless) Madelung constant of the ith ion:
    \begin{equation}
\mathrm{M}_{\mathrm{i}}=\sum_{\mathrm{j}} \frac{\mathrm{z}_{\mathrm{j}}}{\mathrm{r}_{\mathrm{ij}} / \mathrm{r}_0}
\end{equation}
\\
NaCl being an ionic crystal has two separate Madelung constants - one for Na and the other for Cl. Due to the symmetry, their magnitude is identical, and they differ only in sign. The electrical charge of Na+ and Cl- ions is assumed to be +1 and -1, respectively, with zNa = 1 and zCl = -1. The nearest neighbor distance is half the lattice constant of the cubic unit cell. If the ion is sitting at the point (j, k, l) of the lattice, the Madelung constants will then changed to another form:

\begin{equation}\label{eq:Madelung_constant}
M_{\mathrm{Na}}=-M_{\mathrm{Cl}}=\sum_{j, k, \ell=-\infty}^{\infty} \prime \frac{(-1)^{j+k+\ell}}{\sqrt{j^2+k^2+\ell^2}}
\end{equation}
    where the index 0 means that the sum does not contain the term which
    corresponds to (j, k, l) = 0.

The feature of series (3) is that the decay rate ofthe terms is not strong enough to provide absolute convergence(for instance, for the line j = k = l to get infinite harmonic sum).So they are often only conditionally convergent and their convergence/divergence strongly depends on the method of summation.There are two typical methods of summing this series, by expanding cubes or expanding spheres which will be discussed in the next section. 

\end{document}